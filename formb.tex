% -*- coding: UTF-8 -*-
\documentclass[a4paper,11pt]{examdesign}
\usepackage{amssymb,amsmath}
\usepackage[range-phrase = \text{\~{}}, range-units = single,
            binary-units]{siunitx}
\usepackage[scale={.8,.85},footskip=40pt]{geometry}
\usepackage{lastpage,float}
\NumberOfVersions{1}
\usepackage[fullfamily,opticals,swash,minionint,openg,lf]{MinionPro}
\usepackage{mathspec}
% \usepackage{unicode-math}
\usepackage{xeCJK}
\setmainfont{Minion Pro}
\setsansfont{Myriad Pro}
\setmonofont{Consolas}
\sisetup{
    math-micro = \text{μ},
    text-micro = μ,
    math-ohm = \text{Ω},
    text-ohm = Ω,
}
% \setmathfont{XITS Math}
% \setmathfont[range=\mathup/{num,latin,Latin,greek,Greek}]{Minion Pro}
% \setmathfont[range=\mathbfup/{num,latin,Latin,greek,Greek}]{MinionPro-Bold}
% % \setmathfont[range=\mathit/{num,latin,Latin,greek,Greek}]{MinionPro-It}
% \setmathfont[range=\mathbfit/{num,latin,Latin,greek,Greek}]{MinionPro-BoldIt}
% \setmathfont[range=\mathscr,StylisticSet={1}]{XITS Math}
% \setmathfont[range={"005B,"005D,"0028,"0029,"007B,"007D,"2211,"002F,"2215 } ]{Latin Modern Math} % brackets, sum, /
% % \setmathfont[range={"002B,"002D,"003A-"003E} ]{MnSymbol} % + - < = >
% \setmathfont[range={"1D454}]{Latin Modern Math} % openg
% % \setmathrm{Minion Pro}
\setCJKmainfont[BoldFont={FZSongHei-B07S},
              ItalicFont={FZKaiTi},
              SlantedFont={FZFangSongTi}]{FZNewShuSong-Z10}
\setCJKsansfont[BoldFont={FZDaHei-B02S},
              ItalicFont={FZLiShu II-S06S},
              SlantedFont={FZCuYuan-M03S}]{FZHeiTi}
\setCJKmonofont[BoldFont={FZZhongDengXian-Z07S},
              ItalicFont={FZXiYuan-M01S},
              SlantedFont={FZBaoSong-Z04S}]{FZXiDengXian-Z06S}
\defaultfontfeatures{Ligatures=TeX,Scale=MatchLowercase}
\usepackage{graphicx}
\usepackage{tikz}
%\usepackage{floatflt}
\usepackage{array}

\setlength{\parindent}{0pt}
% \setlength{\parskip}{6pt plus 2pt minus 1pt}
\linespread{1.3}
\usepackage{enumitem}
\setlist{itemsep=0.5\itemsep, parsep=0.5\parsep,
         partopsep=0.5\partopsep, topsep=0.5\topsep}

\def\Varangle#1{\kern.75pt\vtop{\hbox{\kern-.75pt$/{#1}$}\kern-.35pt\hrule}}

\begin{document}
\DeclareGraphicsExtensions{.pdf}
\renewcommand\figurename{图}
\newcommand{\bhline}{\noalign{\hrule height 1pt}}
\newcommand{\defen}{\marginpar{\begin{tabular}{!{\vrule width 1pt}c!{\vrule width 1pt}}\bhline 得分\\ \hline \\ \\ \bhline\end{tabular}}}
\newcommand\AnswerLeading{解}
\SectionPrefix{}
\NoKey
\NoRearrange
\ShortKey
\ProportionalBlanks{1.5}
\SectionFont{\large\bf}
\examname{电子技术基础实验}
\DefineAnswerWrapper{\begin{description}\item [\AnswerLeading:]}{\end{description}}
\def\namedata{\large 学号\underline{\hspace{121pt}}姓名\underline{\hspace{98pt}}成绩\underline{\hspace{98pt}}}
\class{\large 学院(部)\underline{\hspace{98pt}}年级\underline{\hspace{98pt}}专业\underline{\hspace{98pt}}}
\begin{examtop}
\setcounter{version}{2}
\begin{center}
\begin{tabular}{r}
    {\Large \bf 苏州大学
    \underline{\hspace{30pt}\examtype\hspace{30pt}} 课程}\hspace{9pt}\medskip \\
    {\Large \bf 期 \underline{\hspace{9pt}末\hspace{9pt}} 试卷 \hspace{17pt}(\textsf{\Alph{version}}) 卷
    \hspace{36pt}共 1 页}\medskip\\
    {\large 考试形式 \underline{\hspace{7pt}开卷\hspace{7pt}}\hspace{48pt}2015 年 12 月}
\end{tabular}
\\\bigskip
\classdata\\ \namedata
\end{center}
% \bigskip
\end{examtop}
\begin{keytop}
\begin{center}
\begin{tabular}{r}
    {\Large \bf 苏州大学
    \underline{\hspace{54pt}\examtype\hspace{54pt}} 课程}\hspace{9pt}\medskip \\
    {\Large \bf \hspace{17pt}期 \underline{\hspace{9pt}末\hspace{9pt}} 考试参考答案\hspace{12pt}共 \pageref{LastPage} 页}\medskip\\
    {\large考试形式 \underline{\hspace{7pt}开卷\hspace{7pt}}\hspace{48pt}2015 年 12 月}
\end{tabular}
\end{center}
\bigskip
\end{keytop}

\newcommand\mathdot[1]{\dot#1}

\begin{shortanswer} % [title={计算题}]

说明:1, 2两题自选其中一题。制作完成并通过测试
后给出成绩,实验板留下。所有测量相关数据都要记录。

成绩评定:功能完成70/80分,线路板焊接质量、走线合理、数据记录等20分。

\begin{question}
    设计移相电路,输入正弦信号。要求输出信号与输入信号相位差
    ±\SI{20}{\degree},±\SI{60}{\degree},用示波器观察。可以改变电路参数
    或改变输入信号频率。要求估算截止频率,相位超前和滞后都要实现。记录
    器件参数、输入输出信号的波形、频率、幅度、相位差等。在
    获得要求的相位差后,切换到方波,观察并记录有关波形和参数,指出是否
    有微分或积分效果,为什么?(90分)
\end{question}

\begin{question}
    放大电路按下图接线,$V_\mathrm{i}$ 输入 \SIrange{0.5}{50}{kHz},
    $V_\mathrm{pp}=\SIrange{1}{2}{V}$ 的正弦信号(5个输入频率,应尽量覆盖
    前述频率范围),记录各器件参数,$V_\mathrm{i}$, $V_\mathrm{o}$ 的波形,
记录频率、幅度并比较两者相位。其中A1: OP07,$R_1 =
R_2$,取 \SIrange{1}{10}{k\ohm}。$R_3=kR_1$,其中 $k\in[2,\,3]$,注意运
放的电源接 ±\SI{6}{V}。维持输入信号幅度不变,提高输入频率,当
$V_\mathrm{o}$ 下降为 \SI{0.5}{kHz} 输出值的0.707倍时,记录对应的上限频
率 $f_\mathrm{H}$,该值有何意义?在该频率下切换到方波输入,观察有何不同,
记录波形和幅度。(100分)
    \begin{figure}[H]
        \hfill\input{fig02}
    \end{figure}
\end{question}

\end{shortanswer}
\end{document}
